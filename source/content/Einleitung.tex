
\chapter{Einleitung}

Dieses Kapitel soll den Leser auf den Inhalt der Arbeit aufmerksam 
machen, ihn mit der Aufgabestellung vertraut machen und "uber die 
Strukturierung und Zielsetzung der Arbeit Auskunft geben.


\section{Motivation}
Objekte werden in der heutigen Tagen immer mehr mit Elektronik und 
Intelligenz versehen. Die Leute wollen aufgrund dieser Entwicklung, dass 
Prozesse oder bestimmte Aufgaben ohne menschliches Eingreifen erledigt 
und miteinander vernetzt werden. Das System soll nur "uberwacht werden 
und die Ergebnisse zu bestimmten Zwecken benutzt werden.

Das Internet der Dinge (in Englisch \textbf{\textit{internet of 
things}}, Kurzform \textbf{IoT}) wird dazu benutzt, um die 
Interaktionzwischen Menschen und vernetzten elektronischen Ger"aten zu 
vereinfachen. 


\section{Aufgabenstellung und Zielsetzung}
Man m"ochte Daten wie Energieverbrauch eines Hauses, Bewegung eines 
Objekts oder die Temperatur eines Raums kennen und "uber lange Strecken 
(20Km) "ubertagen ohne hohe kosten mit aber hohe Batterielaufzeit. Es 
gibt heutzutage Technologien wir GSM, Bluetooth oder Wifi, die diese 
Arbeit erledigen k"onnen. Das Problem dabei ist, dass es beim Nutzen von 
GSM hohe Lizenzkosten fallen mussen. Was Bluetooth und Wifi betrifft, 
sind ihre Reichweite sehr kurz. Dieser Ziel kann mit Hilfe der 
LoRa-Technologie erreicht werden, da sie diese Nachteilen beseitigt.  

In dieser Abschlussarbeit soll ein Prototyp gebaut werden, der mit Hilfe
der LoRa-Technologie Daten an einem Server senden und von diesem Server 
Daten Empfangen. Anders gesagt, diese Bachelorarbeit besch"aftigt sich 
mit der Entwicklung eines vernetzten Systems bestehend aus einem 
3D-Beschleunigungssensor, einem 3D-Gyroskop sowie einem Tem\-peratur- 
und Feuchtigkeitssensor. Die Sensoren messen Daten und "ubergeben diese 
an den STM32L475 Mikrocontroller.

Der Mikrocontroller soll die Daten verarbeiten und mit Hilfe eines 
LoRa-Moduls\cite{AT_Command} drahtlos an einem Server "ubertragen. Bevor 
die "Ubertragung erfolgt, muss das LoRa-Endger"at Zugang zu dem Neztwerk 
durch dem Server bekommen. Nach dem das LoRa-Endger"at dem Netzwerk 
hinzugef"ugt wurde, k"onnen nun Informationen zwischen dem 
LoRa-Endger"at und dem Netzwerk-Server bis zu einem Anwendungsserver 
vertauscht werden. Abbildungen \ref{fig:LRWAN} und 
\ref{fig:LabcsmartLoRawan} geben einen "Uberblick "uber den Aufbau des 
gesamten Systems.

\vspace{2cm}
\begin{figure}[h]
	\centering
	\includegraphics[width=15cm]{source/images/LoRaWAN_NET}
	\caption{Allgemeine LoRaWAN Netzwerkarchitektur 
	\cite{LoRaWAN}\label{fig:LRWAN}}
\end{figure}


\begin{figure}[h]
	\centering
	\includegraphics[width=15cm]{source/images/Gesamtsystem}
	\caption{Labcsmart LoRaWAN Netzwerk \label{fig:LabcsmartLoRawan}}
\end{figure}

\begin{description}
	\newpage
	\item[LoRa:] ist eine abk"urzung f"ur \textbf{\textit{Long Range}} 
	und es ist eine drahtlose Technologie, das geringe Sendeleistung 
	verbraucht wird, um kleine Datenpakete (0,3 Kbps bis 5,5 Kbps) "uber 
	eine lange Strecke zu senden oder zu empfangen.    
	
	\item[End Node Endger"at:] ist ein Ger"at, das aus zwei Teilen 
	Besteht. Ein Funkmodul mit Antenne und einem Mikrocontroller zur 
	Verarbeitung der Daten wie Sensordaten. Diese Daten k"onnen entweder 
	an einem anderen LoRa-Node per Point-To-Point-Verbindung oder an 
	einem LoRaWAN-Netzwerk versand werden.

	\item[LoRaWAN:] steht f"ur \textbf{\textit{Long Range Wide Area 
	Network}} und ist das Kommunicationsprotokol f"ur den Netzwerk.
	
	\item[Gateway:] ist ein Ger"at, das aus mindestens einem 
	Funkkonzentrator, einem Host und einer Netzverbindung zum Internet 
	oder einen privaten Netzwerk (Ethernet, 3G, Wifi), m"oglicherweise 
	einem GPS-Empf"anger besteht.
	
	\item[LoRaWAN Server:] ist ein abstrakter Computer, der die von dem 
	Gateway empfangene RF-Pakete verarbeitet und sendet RF-Pakete als 
	Antwort, dass das Gateway zur"ucksenden muss.
	\vspace{1cm}
	\item[Application Server:] ist eine Anwendung, womit der Benutzer 
	die von den Sensoren gemesennen Daten entweder Tabelarisch oder 
	Grafisch ansehen kann.
	
	\item[Uplink:] ist die Kommunikation von einem Endger"at zu einem 
	Gateway. 
	
	\item[Downlink:] ist die Kommunikations von einem Gateway zu einem 
	Endger"at.
\end{description}


\section{Gliederung der Arbeit}

Das Kapitel \ref{Komponente} gibt einen detaillierten "Uberblick "uber 
allen Hardware-Komponenten, die bei der Entwicklung eines Endger"at 
verwendet werden. Als Erstes wird auf die Eigenschaften von des 
benutzten STM32-Nucleo Board eingegangen. Diesem Kapitel ist auch zu 
entnehmen, warum genau dieses Board gew"ahlt wurde. 

Als n"achstes wird das LoRa-Modul. Dieses LoRa-Modul wird dazu 
verwendet, um die gesammelten Daten dem Server drahtlos zu "ubertragen. 
Dieser Kapitel berichtet "uber das Funkprotokol, das zur "Ubertragung 
der Daten eingesetzt wurde, und wie diese Daten gerichert werden.
 
Das Kapitel \ref{G_S} beschreibt den LoRaWAN-Server und das Gateway, 
zwei wichtige Teile dieser Thesis. Die Funktionsweise wird erkl"art und  
die Servereinstellung wird gezeigt. Diese Einstellung sind notwendig, 
weil sie erlauben einem Endger"at denḿ Netzwerk einzutreten.

Als n"achstes wird die Softwareentwicklung behandelt. Hier geht es 
erstmal um die Entwicklungsumgebung des gesamten Projekts (Eclipse). 
Anschlie\ss{}end daran werden die angewante Bibliotheken dargestellt, 
ihre Installation und Nutzung erkl"art. Es wurde f"ur diese Arbeit zwei 
bekannte Kommunikationsschnittstellen verwendet (I2C und UART) Sie 
erfahren auch wie diese Schnittstellen mit der Programmiersprache C 
angesteuert werden, und welche Software-Tricks eingesetzt wurden, um 
AT-Befehle zu senden.

Anschlie\ss{}end  wird im Kapitel \ref{Fazit} eine Zusammenfassung und 
einen kleinen Ausblick der Arbeit gegeben.
   