\chapter{Zusammenfassung}\label{Fazit}

F"ur das Ziel dieser Thesis war ich daf"ur 
verantwortlich, ein IoT-Endger"at basierend auf der 
LoRa-Technologie mit einem STM32-Mikrocontroller zu 
entwickeln. Diese IoT-Endger"at erfasst und sammelt 
sowohl Umwelt-Daten, wie Temperatur und Feuchtigkeit, 
als auch die Beschleunigung. Diese Daten werden "uber 
das LoRaWAN-Protokoll an einem Embedded-Linux basiertes 
Gateway gesendet, entweder lokal in diesem Gateway 
verarbeitet oder diesmal "uber das Internet an einem 
Anwendungsserver weitergeleitet. 

Das IoT-Endger"at kann abh"angig des Zwecks 
verschiedene Daten erfassen. Das STM32L4-discovery Kit 
wurde ausgew"ahlt weil es "uber einen eingebetteten 
Debugger/Programmierer, eingebettete Sensoren verf"ugt, 
und leicht erweiterbar ist. Es gibt zwar viele 
LoRa-Module aber allen passen nicht zu der zu 
realisierenden Anwendung. Das i-nucleo-lrwan1 war 
daf"ur geeignet, weil es sich durch UART ansteuern 
l"asst.

\section{Ausblick}

Es k"onnen weitere Endger"ate an dem Netzwerk 
hinzugef"ugt werden. Diese soll einfach nur wie das 
Labcsmart LoRa-endger"at \ref{fig:loranode} 
konfiguriert werden. Die Sensoren k"onnen angepasst 
werden, f"ur die Messung des Stromverbrauchs zum 
Beispiel. Es gibt auch m"oglichkeiten die gemessene 
Daten graphisch darzustellen. 

Dazu muss der MQTT-Broker mit einem Dashboard 
(Open-source, um die Kosten klein wie m"oglich zu 
halten) wie \textbf{Thinkboard} verbunden und 
eingestellt werden. Solche Anwendungen sind 
User-Freundlicher als eine \ac{CSV}-Datei oder die 
Frames, sie in einem Server zu beobachten sind.

\section{Fazit}

Diese Bachelorarbeit war eine gro\ss{}e 
Herausforderung, vor allem wegen der Entscheidung diese 
im Ausland zu absolvieren. Ich konnte viele neue 
Technologien Kennenlernen und diese Praktisch einsetzen. 
Diese Auslandserfahrung  hat mir trotz 
COVID-19-Pendemie sehr gut gefallen. Ich die 
M"oglichkeit Aufgaben selbst"andig zu "ubernehmen, um 
mich praktisch tiefergehend mit den T"atigkeiten 
vertraut zu machen.

Ich habe meine Zeit in Frankreich sehr genossen, da mir 
die Aufgaben sehr gut gefallen haben. Das entspannte 
Arbeitsklima zwischen meinem Chef und mir hatte darauf 
einen gr\ss{}en Einfluss, weil er bei Fragen oder 
Probleme immer verf"ugbar war. Seine Kritiken waren 
auch sehr konstruktiv.  Ich bereue meine Entscheidung, 
im Ausland zu gehen, auf keinen Fall.    
