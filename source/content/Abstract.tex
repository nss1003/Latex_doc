
\chapter{ Abstract} \label{Abstract}	
\enquote{Die Ziele des IoT sind die IT-Vernetzung von Gegenst"anden und 
die Bereitstellung von funktionalit"aten Beziehungsweise 
Dienstleistungen, wie es so noch nie gab.} Das Internet der Dinge ist 
eine zentrale Bedeutung f"ur das Privatleben und die Wirtschaft geworden, 
da ein riesiger Anteil an vernetzten Objekten miteinander kommunizieren. 
Es entstehen neu Einsatzgebiete und Anwendungen mit dem Einsatz 
hochmoderner Technologien. 

Durch den sofortigen Zugriff auf Informationen "uber die Umwelt und die 
Objekte erh"ohen sich Effizienz und Produktivit"at, wodurch sich gro\ss{}e 
Chancen f"ur due Wirtschaft und das Privatleben er"offnen. Der Einsatz des 
IoT erm"oglicht es, umfangreiche Echtzeitinformationen aus der Umwelt 
oder  vom beweglichen und bewegungslose Objekte zu ber"ucksichtigen. Die 
Vorteile sind unter anderen, die zeitliche Verfolgung von Gegenst"ande.

Diese Arbeit besch"aftig sich mit der Entwicklung eines IoT-Endger"at, das 
auf der LoRa-Technilogie und einem STM32L4-Mikrocontroller basiert ist. 
Diese IoT-Endger"at erfasst und sammelt sowohl Umwelt-Daten, wie
Temperatur und Feuchtigkeit, als auch die Beschleunigung. Diese Daten 
werden "uber das LoRa-WAN-Protokol an einem Embedded-Linux basiertes 
Gateway gesendet, entweder lokal in diesem Gateway verarbeitet oder 
diesmal "uber das Internet an einem Anwendungsserver weitergeleitet. 


\newpage




